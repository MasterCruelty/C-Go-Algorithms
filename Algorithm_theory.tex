\documentclass{article}
\usepackage{amsmath} % import of math elements
%\usepackage{mathtools} %import of other math elements
\usepackage{tikz} 
\usetikzlibrary{shapes,positioning,calc} 

%-------------------------------------------------------
% Document information
%-------------------------------------------------------

\title{Algorithm and data structures} %Title 

\author{Roberto \textsc{Antoniello}} %author name

\begin{document}
\maketitle % show the title and author and date
%-------------------------------------------------------
%Introduction
%-------------------------------------------------------

\begin{center} In this file I will resume all the concepts I liked while studying the course of Algorithm and data structures.\end{center}

\section{Binary search}
This algorithm can be used only if you have a sorted array. Here how it works: \\
BinarySearch take a sorted array as input and return an index as output. So it returns the index of the found element or -1 if not found.\\
When it starts execution, the algorithm save three variables sx, dx and m. The "m" variable is the index in the middle of the array, sx and dx are the first and the last index. It asks if the element is less or more than the element in m position. \\
Basically, if the element is x the question is: 
x < A[m]  or  x > A[m] ? 
We are reducing the search space by 2 every time because if it's less, our "dx" become "m", otherwise our "sx" become "m+1". \\
A the first iteration the search space is n elements. At the second iteration it is $\cfrac{n}{2}$. A the third one it is $\cfrac{n}{2^2}$ and so on.\\
At the i° iteration it will be $\cfrac{n}{2^i}$.\\
During the last iteration the size of our array A is 1. So:\\
$\cfrac{n}{2^i} = 1 \Rightarrow n = 2^i \Rightarrow i = \log_2{n}$ \\
We have just said the amount of steps are $\log_2{n} \Rightarrow O(\log{n})$

\end{document}
